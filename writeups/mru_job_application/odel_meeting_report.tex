\documentclass[11pt,a4paper]{article}

% Packages
\usepackage[left=2.5cm, right=2.5cm, top=2.5cm, bottom=2.5cm]{geometry}
\usepackage[T1]{fontenc}
\usepackage{graphicx}
\usepackage{xcolor}
\usepackage{hyperref}
\usepackage{parskip}
\usepackage{enumitem}
\usepackage{booktabs}
\usepackage{fancyhdr}
\usepackage{titlesec}
\usepackage{tabularx}

% Colors
\definecolor{primary}{RGB}{0,82,147}
\definecolor{darkgray}{RGB}{60,60,60}
\definecolor{lightgray}{RGB}{240,240,240}

% Hyperlink setup
\hypersetup{colorlinks=true, linkcolor=primary, urlcolor=primary}

% Section formatting
\titleformat{\section}{\large\bfseries\color{primary}}{}{0em}{}[\color{primary}\titlerule]
\titleformat{\subsection}{\normalsize\bfseries\color{darkgray}}{}{0em}{}
\titlespacing{\section}{0pt}{14pt}{6pt}
\titlespacing{\subsection}{0pt}{10pt}{4pt}

% Header/Footer
\pagestyle{fancy}
\fancyhf{}
\renewcommand{\headrulewidth}{0pt}
\fancyfoot[C]{\small\color{darkgray} Muteesa I Royal University --- ODEL Stakeholder Meeting Report \quad | \quad Page \thepage}

\begin{document}

% ---- TITLE PAGE / HEADER ----
\begin{center}
\includegraphics[height=2.5cm]{mru_logo.png}\\[6pt]
{\color{primary}\rule{\linewidth}{1pt}}\\[10pt]
{\LARGE\bfseries\color{primary} Report on the Open, Distance and\\[2pt] e-Learning (ODEL) Stakeholder Meeting}\\[10pt]
{\color{primary}\rule{\linewidth}{1pt}}\\[12pt]

\begin{tabular}{r l}
\textbf{Date of Meeting:} & 10th February 2026 \\
\textbf{Institution:} & Muteesa I Royal University (MRU) \\
\textbf{Submitted to:} & The Vice Chancellor, MRU \\
\end{tabular}
\end{center}

\vspace{10pt}

% ---- 1. INTRODUCTION ----
\section{Introduction}

This report summarises the proceedings and key outcomes of the Open, Distance and e-Learning (ODEL) stakeholder meeting held on 10th February 2026. The meeting brought together representatives from over 51 universities across Uganda, along with an external technology partner, to discuss the progress, challenges, and future direction of ODEL implementation across participating institutions.

As my first assignment at Muteesa I Royal University, I attended this meeting to observe, take notes, and report on the discussions for the attention of the Vice Chancellor.

% ---- 2. BACKGROUND ----
\section{Background: ODEL in Uganda}

Open, Distance and e-Learning (ODEL) refers to a mode of education delivery that allows learners to study flexibly---remotely and at their own pace---using digital tools, online platforms, and blended learning approaches. In Uganda, ODEL is regulated by the \textbf{National Council for Higher Education (NCHE)}, which issued specific guidelines that universities must follow to offer accredited distance learning programmes.

The adoption of ODEL gained significant momentum during and after the COVID-19 pandemic, when universities were forced to shift to online delivery. However, even before the pandemic, institutions such as Makerere University and Mbarara University of Science and Technology had been experimenting with distance learning models. Key enablers include:

\begin{itemize}[leftmargin=16pt, itemsep=2pt]
    \item \textbf{RENU} (Research and Education Network for Uganda) --- Provides high-speed internet connectivity to over 1,000 campuses across Uganda and offers cloud, eduroam, and colocation services to member institutions.
    \item \textbf{Moodle} --- An open-source Learning Management System (LMS) widely adopted by Ugandan universities for course delivery, assessments, and student engagement.
    \item \textbf{BigBlueButton} --- An open-source web conferencing tool integrated with Moodle to facilitate live virtual classes.
    \item \textbf{Astria Learning} --- A technology partner (headquartered in Tampa, FL, USA) that provides eCampus solutions and LMS platforms to African universities, supporting digital transformation.
\end{itemize}

Despite these enablers, ODEL adoption in Uganda continues to face significant barriers, particularly around internet infrastructure, device access, human resource capacity, and institutional policy alignment.

% ---- 3. MEETING PARTICIPANTS ----
\section{Participating Institutions and Representatives}

The following institutions and individuals were represented at the meeting:

\begin{table}[h!]
\centering
\begin{tabularx}{\textwidth}{l X}
\toprule
\textbf{Institution / Organisation} & \textbf{Representative / Notes} \\
\midrule
Mbarara University of Science \& Technology & Representative present \\
Soroti University & Justas (representative) \\
Busitema University & Leonard (representative) \\
Clarke International University (CIU) & Prof.\ Robert \\
Mountains of the Moon University (MMU) & Representative present \\
Astria Learning & Jeff Bordes \\
\bottomrule
\end{tabularx}
\end{table}

% ---- 4. SUMMARY OF DISCUSSIONS ----
\section{Summary of Discussions}

The meeting was structured around several thematic areas. Below is a summary of the key points raised under each.

\subsection{4.1\quad Opening Remarks}

The representative from \textbf{Mbarara University} opened by appreciating the efforts of the organising team in convening the meeting and providing updates on various ODEL projects across institutions.

\subsection{4.2\quad Human Resources and Capacity Building}

\begin{itemize}[leftmargin=16pt, itemsep=2pt]
    \item \textbf{Soroti University} highlighted that the main challenges facing ODEL adoption at their institution revolve around \textit{human resources, infrastructure, and market access}. They specifically requested capacity building in digital marketing, digital content creation, and content distribution.
    \item \textbf{Prof.\ Robert (CIU)} raised an important concern: that many ODEL programmes focus on developing human resources at the administrative level but neglect to adequately train \textit{lecturers and students} on how to actually use the platforms. He noted that CIU does train its users, but emphasised the need for \textit{continuous and ongoing training} to ensure effective utilisation of the LMS.
\end{itemize}

\subsection{4.3\quad Communication and Policy}

\begin{itemize}[leftmargin=16pt, itemsep=2pt]
    \item \textbf{Busitema University} stressed that simple and clear communication is essential for engaging stakeholders effectively. They also flagged that there are \textit{conflicts between ODEL practices and existing institutional policies}, which need to be resolved for smoother implementation.
\end{itemize}

\subsection{4.4\quad Infrastructure Challenges}

This was the most extensively discussed topic. Several institutions shared their experiences:

\begin{itemize}[leftmargin=16pt, itemsep=2pt]
    \item \textbf{Mountains of the Moon University (MMU)} reported that they use \textbf{Moodle} and \textbf{BigBlueButton} for online learning. To support ODEL, MMU procured dedicated bandwidth from \textbf{RENU}. However, they still face persistent challenges with internet connectivity, especially for students in remote areas. When large numbers of students are online simultaneously, the network becomes strained. Access to devices also remains a constraint.
    \item \textbf{Justas (Soroti University)} reported limited internet connectivity and noted that the cost of internet is prohibitively high for students. He requested that government consider subsidising internet costs and invest in improving infrastructure in rural areas. He suggested that ODEL systems should be hosted on the cloud for reliability and accessibility, while acknowledging the cost implications. He also raised concerns about data storage and backup, noting the risk of losing important academic records.
    \item \textbf{Jeff Bordes (Astria Learning)} appreciated the candid feedback from the universities and emphasised that addressing infrastructure challenges is fundamental to the long-term success of ODEL in the region.
\end{itemize}

\subsection{4.5\quad Technical and Market Access Issues}

\begin{itemize}[leftmargin=16pt, itemsep=2pt]
    \item \textbf{Leonard (Busitema University)} reported that their LMS frequently \textit{freezes during quiz and assessment sessions}, causing frustration for both students and lecturers. He attributed this to limited bandwidth and the fact that many student devices are unable to handle the demands of the system, leading to technical disruptions during learning activities.
\end{itemize}

% ---- 5. KEY THEMES ----
\section{Key Themes and Observations}

From the discussions, the following recurring themes emerged:

\begin{enumerate}[leftmargin=16pt, itemsep=2pt]
    \item \textbf{Internet Connectivity} remains the single biggest barrier. Even institutions that have procured RENU bandwidth still struggle, particularly during peak usage and in remote areas.
    \item \textbf{Device Access} is a secondary but significant constraint. Many students lack personal computers or smartphones capable of supporting LMS platforms.
    \item \textbf{Training Gaps} exist at all levels. Institutions tend to invest in platform setup but underinvest in training lecturers and students to use those platforms effectively.
    \item \textbf{Policy Alignment} is needed. Existing university policies were not designed with ODEL in mind, creating friction in implementation.
    \item \textbf{System Reliability} is a concern. LMS platforms freeze under load, and data backup mechanisms are inadequate.
    \item \textbf{Cost} is a barrier for both institutions (cloud hosting, bandwidth) and students (personal internet access).
\end{enumerate}

% ---- 6. RECOMMENDATIONS ----
\section{Recommendations}

Based on the discussions, the following actions are recommended for MRU's consideration:

\begin{enumerate}[leftmargin=16pt, itemsep=2pt]
    \item \textbf{Invest in RENU membership and bandwidth} to ensure reliable campus connectivity for any future ODEL rollout at MRU.
    \item \textbf{Develop a structured training programme} for both lecturers and students before deploying any LMS, rather than assuming users will figure it out on their own.
    \item \textbf{Review and align institutional policies} to accommodate ODEL delivery modes, assessment methods, and quality assurance requirements as stipulated by NCHE.
    \item \textbf{Explore cloud-hosted LMS solutions} (such as those offered by Astria Learning or Moodle Cloud) to reduce the burden of on-premise server management and improve system reliability.
    \item \textbf{Engage with peer institutions} (MMU, Busitema, CIU) to learn from their experiences and avoid repeating the same challenges.
    \item \textbf{Advocate for government subsidies} on internet access for students, particularly those in rural and underserved areas.
\end{enumerate}

% ---- 7. CONCLUSION ----
\section{Conclusion}

The ODEL stakeholder meeting provided a valuable overview of the current state of open and distance learning in Ugandan universities. While progress has been made---particularly in the adoption of platforms like Moodle and BigBlueButton---significant challenges around internet infrastructure, device access, user training, and policy alignment persist. These are not unique to any single institution; they are systemic issues that require coordinated effort across universities, government, and technology partners.

For MRU, the meeting offers useful lessons as the university considers its own ODEL strategy. Investing early in infrastructure, training, and policy alignment will be critical to avoiding the pitfalls that other institutions have encountered.

\vspace{16pt}

% ---- SIGNATURE ----
\noindent{\color{primary}\rule{\linewidth}{0.4pt}}
\vspace{6pt}

\noindent\textbf{Prepared by:}\\[4pt]
\includegraphics[height=1.2cm]{muhindo_signature.png}\\[2pt]
\textbf{Muhindo Mubaraka}\\
Manager, Information Systems\\
Muteesa I Royal University\\[4pt]
\textbf{Date:} \today

\end{document}
